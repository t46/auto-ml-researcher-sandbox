
\documentclass{article}

\usepackage{neurips_2023}
\usepackage[utf8]{inputenc}
\usepackage[T1]{fontenc}
\usepackage{hyperref}
\usepackage{url}
\usepackage{booktabs}
\usepackage{amsfonts}
\usepackage{nicefrac}
\usepackage{microtype}
\usepackage{xcolor}

\title{Title}

\begin{document}
\maketitle

\begin{abstract}
  {abstract-description}
  % Write abstract here.
\end{abstract}

\section{Introduction}
{introduction-description}
% Write introduction here.

\section{Proposed Method}
{method-description}
% Short description of the proposed method in a paragraph.

\section{Experiment}
% Short description of the experiment in a paragraph.

\subsection{Experiment Setup}
{experiment-design-summary}

\paragraph{Dataset}
{dataset-description}
% When using a dataset for the experiment, describe the dataset used here. If modifications were made to an existing dataset or a new dataset was created specifically for this experiment, describe that as well here.

\paragraph{Model}
{model-description}
% Describe the model used foar the experiment here.

\paragraph{Training}
If the experiment involved training a model, describe the training process here.

\paragraph{Evaluation}
Describe the evaluation process here.

\paragraph{Baseline}
If the experiment involves a comparison with existing methods or a control group, describe here the methods you are comparing the proposed method against.

\subsection{Results}
{experiment-results}
% Describe the results of the experiment here.

\paragraph{Comparison with Existing Methods}
If the experiment involves a comparison with existing methods or a control group, describe the comparison here.

\paragraph{Ablation Study}

\paragraph{Error Analysis}

\section{Conclusion}
{conclusion-description}


\end{document}
